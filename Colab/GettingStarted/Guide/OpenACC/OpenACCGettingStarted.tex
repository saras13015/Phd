\documentclass[10pt,a4paper]{article}
\usepackage[utf8]{inputenc}
\usepackage{amsmath}
\usepackage{amsfonts}
\usepackage{amssymb}
\usepackage{hyperref}
\usepackage{listings}
\usepackage[many]{tcolorbox}
\tcbuselibrary{listings}

\newtcblisting{mylisting}{
  listing only,
  hbox,
  colframe=cyan,
  colback=cyan!10,
  listing options={
    basicstyle=\small\ttfamily,
    breaklines=true,
    columns=fullflexible
  },
}

%hyperlink parameters
\hypersetup{
    colorlinks=true,
    linkcolor=blue,
    filecolor=magenta,      
    urlcolor=cyan,
    pdftitle={Overleaf Example},
    pdfpagemode=FullScreen,
    }
\urlstyle{same}

\author{Sarah}
\title{OpenACC using Colab}
%\date{today}

\begin{document}
\maketitle{}
\newpage

\section{Part 1: Introduction}
This document offers an introduction to porting algorithms on GPU with OpenACC using Colab.
\section{Part 2: Running a Fortran code in Colab}
\subsection{About Colab}


\subsubsection{Compilation and Execution}

\subsubsection{Preview on the CPU code}


\subsubsection{OpenACC}



\subsection{Compiling \& Executing}
In our particular case, it is relevant now to introduce Fortran. Because it is a high-level computer langage and more important it is a compiled langage. That means it cannot be run until the compilation stage.\\
In the other hand, we work in the Jupyter Notebook which is an interactive environment and interpreted oriented.\\
A way to overcome this incompatibility is to write the Fortran code inside a file, then, to execute the file as a script.
To do so, the fortran kernel is no longer required, and we need to switch to the python kernel to use what we call Magics. They are powerful tools provided by the IPython kernel.
\begin{itemize}

\item
\begin{lstlisting}[language=bash]
%%writefile
\end{lstlisting}
Writes the content of the cell into a file that should be specify right after the magics command
\begin{mylisting}
%%writefile get_age.f95

program get_age
    real :: year, age
    print *, 'What year were you born?'
    read *, year
    age = 2022 - year
    print *, 'Your age is', age
end program get_age
\end{mylisting}
\item 
\begin{lstlisting}[language=bash]
%%bash
\end{lstlisting}
Allow us to run cells with bash.
\begin{mylisting}
%%bash

gfortran -ffree-form get_age.f95
./a.out
1984
\end{mylisting}
\end{itemize}
\subsection{Results}

\section{Part 2: OpenACC on Colab}


\end{document}