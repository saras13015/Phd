\documentclass[10pt,a4paper]{article}
\usepackage[utf8]{inputenc}
\usepackage{amsmath}
\usepackage{amsfonts}
\usepackage{amssymb}
\usepackage{hyperref}
\usepackage{listings}
\usepackage[many]{tcolorbox}
\tcbuselibrary{listings}

\newtcblisting{mylisting}{
  listing only,
  hbox,
  colframe=cyan,
  colback=cyan!10,
  listing options={
    language=Fortran,
    basicstyle=\small\ttfamily,
    breaklines=true,
    columns=fullflexible
  },
}

%hyperlink parameters
\hypersetup{
    colorlinks=true,
    linkcolor=blue,
    filecolor=magenta,      
    urlcolor=cyan,
    pdftitle={Overleaf Example},
    pdfpagemode=FullScreen,
    }
\urlstyle{same}

% Code style font
\usepackage{xcolor}
\definecolor{light-gray}{gray}{0.95}
\newcommand{\code}[1]{\colorbox{light-gray}{\texttt{#1}}}
%\newcommand{\code}{\texttt}

% Fortran source code style



\author{Sarah}
\title{Paper reviews}
%\date{today}

\begin{document}
\maketitle{}
\newpage

\section{URANOS: a GPU accelerated Navier-Stokes solver for compressible wall-bounded flows}
\subsection{Context}
\begin{itemize}
\item Solver for high-fidelity modeling of comopressible wall flows
\item Massively parallel GPU-accelerated
\item Based on modern high\-fidelity and high-resolution discretization strategies for time-accurate compressible flow predictions
\item Provides 
\begin{itemize}
\item 6 different convective scheme implementations
\item a cutting-edge method for viscous terms treatment
\item 3 different frameworks for turbulence modeling (DNS, LES, WMLES)
\item A high-order FD approach (from 2nd -> $6^{th}$ order spatial accuracy)
\end{itemize}
\item Combines multiple 3D MPI parallelization strategies with the open standard, OpenACC, f or machine wide, on node, and on GPU parallelism
\end{itemize} 
\subsection{Numerical methods}
\begin{itemize}
\item High-Order FD approach matching for both uniform and non-uniform Cartesian structures
\item 6 different convective schemes
\begin{itemize}
\item A central, zero-dissipative, $6^{th}$ order fully-split convective Energy-Preserving (EP) method to deal primarily with shock-free or smooth flows
\item 3 increasingly high-order WENO mehtods
\item 2 low-dissipative Targeted Essentially Non-Oscillatory (TENO) approaches.
\end{itemize}
\item Shock-capturing method

\end{itemize}


\subsection{Acceleration}
\begin{itemize}
\item GPU porting
\item Different approaches vary in terms of their degrees of portability, adaptability and computational performance
\item Things to consider: the initial cost fo code development \& the long term maintenance costs
\item Better to have a single code base which targets different architectures
\item Rather than programming with vendor-specific languages, the programmer can focus on accelerate in a vendor-neutral manner.
\item The compiler transforms directives into device-specific application code.
\end{itemize}

\subsubsection{OpenACC \& MPI}
\begin{itemize}
\item H2D/D2H data movements reduced as much as possible
\item \code{parallel} construct is preferrred rather than \code{kernels} as it allows the user more control.
\item And it enables controlling the loop granularity through the clauses (loop, ...)
\begin{itemize}
\item Coarse-grained parallelism (\textit{gang})
\item Fine-grained parallelism (\textit{worker})
\item Single Instruction Multiple Data level (\textit{vector})
\end{itemize}
\item \code{collapse} clause allow unifying all the iteration of nested loops in a single one
\end{itemize}

\begin{itemize}
\item MPI: the standard for inter-node data transfers
\item Transition to a multi-GPU logic is not straightforward and hardware-unrelated.
\end{itemize}

\subsection{Discussion}

\section{Glossaire}
\textbf{Time-accurate:} That can provide solutions to the full unsteady equations. The time step is used through the grid.\\
\textbf{GPU-Direct/non-Direct}:




\end{document}